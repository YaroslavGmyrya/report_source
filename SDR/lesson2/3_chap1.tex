\chapter{Лекция}
\label{ch:intro}

На прошлом занятии мы рассматривали архитектуру простого передатчика и то, как в нем формируется и отправляется сигнал.

Сейчас разберем архитектуру приемника и то, как он принимает сигнал и преобразует его в данные.

Сигнал поступает на антенну приемника и первым делом нам нужно выяснить амплитуду sin и cos.
Нужно выделить символы. Для этого подаем сигнал на демодулятор. Внутри него выполняются
следующие действия: перемножает приемный сигнал на несущие сигналы (sin и cos), потом это
подается на фильтр и на выходе получаем I и Q, но уже слегка измененные из-за помех сигнала.
Далее эти I и Q поступают на ацп, который разобьет сигнал I и Q на семплы и в этих отчетах
нужно выполнить символьную синхронизацию - выделить из этих семплов символы, которые потом
нужно подать на демаппер (тот же маппер, но с обратной задачей)
\endinput
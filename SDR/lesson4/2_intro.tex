\chapter{Цель и задачи}
\label{ch:intro}

Удобство сопи сдр - настройка под капотом. Не нужно обращаться к конкретной модели усройства (тх рх, эт разные устройства с точки 
зрения линукс). АЦП внутри сдр работает с разрядностью 12 бит, т.е значения амплитуды -2048-2047. Минимум, в который мы
можем поместить переменную это инт16, ай и ку - разные переменные типа инт 16. Сдр странно хранит данные (со сдвигом видно на 
картинке) и значения уже до 16384. Есть 2 массива - буфер на передачу и прием, они хранят айку семплы, они хранятся вместе 
сначала ай потом ку. Пара таких чисел - семпл. Отсюда следует, что размер буфера должен быть больше вдвое, чем кол-во 
принимаемых семплов. Еще есть варинт класть в инт32 и сдвигами выбирать числа. У нас есть пустой буфер, поссле выозова функции 
ридстрим, то функци полуает данне из сдр и кладет в буффер. После выполнения программы мы можем посомтреть семплы. Чтобы делать
эти действия постоянно, нужно обернуть эту функцию в цикл (while(1)). Таймпштемп - текущее системное время внутри сдр при получении
или отправке данных. Нужен для того, чтобы понять, синхронно ли работает система или нет. К примеру если у нас 1920 семплов
то разница между приемом должна быть 1920*наносекунды.
Работа с передачей
Формирование исходного массива
картинка с доски
Если мы отдалим приемник и передатчик, то сигнал упадет в амплитуде
картинка


Сопи сдр позволяет планировать время, когда будет передан след блок тх. Мы можем сказать: передай семп на 4 мс в будущее.
Это про синхронизацию. Т.е приемник будет знать, что через 4мс ему придет блок, который нужно декодирровать.

\endinput
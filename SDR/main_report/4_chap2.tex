% \chapter{Практическая часть}
% \label{ch:chap2}

% \section*{Adalm Pluto SDR}

% \textbf{Adalm Pluto SDR} --- компактная и автономная портативная SDR-платформа, 
% разработанная компанией \textbf{Analog Devices} для обучения основам SDR и радиочастотных технологий, 
% а также подходящая для радиолюбительских экспериментов. 

% Она сочетает в себе \textbf{приемопередатчик AD9363} и \textbf{процессор Xilinx Zynq}, позволяя генерировать 
% и измерять аналоговые радиочастотные сигналы в широком диапазоне частот. 

% \begin{figure}[H]
%     \centering
%     \includegraphics[width=0.8\textwidth]{AdalmPluto.png}
%     \caption{Внешний вид Adalm Pluto}
% \end{figure}

% \section*{Архитектура Adalm Pluto SDR}

% \begin{figure}[H]
%     \centering
%     \includegraphics[width=0.6\textwidth]{AdalmArch.png}
%     \caption{Архитектура Adalm Pluto}
% \end{figure}

% \section*{Описание основных блоков Adalm Pluto}

% \subsection*{\textbf{PA (Power Amplifier)}}
% \begin{description}
%   \item[\textbf{Функция:}] Усиление сигнала до уровня, требуемого для излучения.
% \end{description}

% \subsection*{\textbf{LNA (Low Noise Amplifier)}}
% \begin{description}
%   \item[\textbf{Функция:}] Усиление слабого приёмного сигнала с минимальным добавлением шума перед его дальнейшей обработкой.
% \end{description}

% \subsection*{\textbf{ADC / DAC}}
% \begin{description}
%   \item[\textbf{ADC (Analog-to-Digital Conversion):}] оцифровка аналогового сигнала в цифровой поток для последующей цифровой обработки.
%   \item[\textbf{DAC (Digital-to-Analog Conversion):}] преобразование цифровых семплов в аналоговый сигнал перед микшированием.
% \end{description}

% \subsection*{\textbf{FIR (Finite Impulse Response)}}
% \begin{description}
%   \item[\textbf{Функция:}] детальная фильтрация, коррекция формы спектра, компенсация искажений.
% \end{description}

% \subsection*{\textbf{Mixer}}
% \begin{description}
%   \item[\textbf{TX:}] перенос низкочастотного сигнала (baseband) на несущую частоту (subcarrier) для отправки в эфир.
%   \item[\textbf{RX:}] перенос высокочастотного сигнала на низкочастотный для дальнейшей обработки.
% \end{description}

% \subsection*{\textbf{Filter}}
% \begin{description}
%   \item[\textbf{TX:}] фильтрация выходного сигнала перед передачей, подавление лишних гармоник.
%   \item[\textbf{RX:}] фильтрация принимаемого сигнала, подавление внеполосных помех перед оцифровкой.
% \end{description}

% \subsection*{\textbf{libiio}}
% \begin{description}
%   \item[\textbf{Описание:}] библиотека, которая облегчает работу с устройствами ввода/вывода в \textbf{Linux}, особенно с радиочипами серии \textbf{AD936x}. 
%   Она даёт \textbf{API} для обмена данными и управления устройствами.
% \end{description}

% \subsection*{\textbf{Linux Kernel}}
% \begin{description}
%   \item[\textbf{Описание:}] ядро \textbf{Linux}, управляющее процессором, памятью и устройствами ввода-вывода.
% \end{description}

% \subsection*{\textbf{DMA (Direct Memory Access)}}
% \begin{description}
%   \item[\textbf{Описание:}] механизм, который автоматически переносит большие объёмы данных между устройством и памятью, разгружая процессор.
% \end{description}

% \subsection*{\textbf{Drivers}}
% \begin{description}
%   \item[\textbf{Описание:}] инструкции для ядра, объясняющие, как правильно пользоваться конкретным оборудованием.
% \end{description}

% \subsection*{\textbf{Xilinx Zynq}}
% \begin{description}
%   \item[\textbf{Описание:}] семейство микросхем от компании Xilinx. На одном кристалле 
%   объединены ARM-процессор (обычно на 2 ядра), который запускает Linux, управляет периферией, и ПЛИС (FPGA) для реализации аппаратных блоков (фильтры, ускорители обработки сигналов) и для более скоростных вычислений в реальном времени. 
% \end{description}

% \subsection*{\textbf{USB 2.0}}
% \begin{description}
%   \item[\textbf{Описание:}] порт для связи хоста и Adalm Pluto.
% \end{description}

% \subsection*{\textbf{GNU Radio}}

% \begin{figure}[H]
%     \centering
%     \includegraphics[width=1.0\textwidth]{GNURadio.png}
%     \caption{Логотип GNU Radio}
% \end{figure}

% \textbf{GNU Radio} --- это инструмент с открытым исходным кодом для разработки программного обеспечения 
% в сфере программно-определяемого радио. 

% Он позволяет при помощи «строительных блоков» создавать конфигурации радиоустройств, не написав ни одной строчки кода, 
% и запускать программы непосредственно с использованием SDR-модулей, например: \textbf{Adalm-Pluto}, \textbf{LimeSDR} и др. \\

% В библиотеке имеется широкий спектр функций для цифровой обработки сигналов. Модули написаны на \textbf{C++}, 
% а их взаимодействие реализовано на \textbf{Python}. Приложения можно строить как через \textbf{API GNU Radio}, 
% так и посредством графического интерфейса \textbf{GNU Radio Companion (GRC)}.

% \section*{\textbf{Построение схемы в GNU Radio}}

% \subsection*{\textbf{Блок options}}

% \begin{figure}[H]
%     \centering
%     \includegraphics[width=1.0\textwidth]{options.png}
%     \caption{Блок options}
% \end{figure}

% Этот блок задает настройки проекта. Самое важное здесь: \textbf{Output Language} и \textbf{Generate Options}. \\
% \textbf{Output Language} — язык, на котором будет сгенерирован код программы (у меня это Python). \\
% \textbf{Generate Options} — используемый графический интерфейс (у меня это QT).

% \subsection*{\textbf{Блок variable}}

% \begin{figure}[H]
%     \centering
%     \includegraphics[width=1.0\textwidth]{var.png}
%     \caption{Блок variable}
% \end{figure}

% В этом блоке можно задать переменную (почти как в языке программирования). Переменная имеет ID (имя) и значение.  
% Я таким образом задаю переменную \texttt{samp\_rate} (частоту дискретизации), равную $2.4 \times 10^6$ Hz.

% \subsection*{\textbf{QT GUI Range}}

% \begin{figure}[H]
%     \centering
%     \includegraphics[width=1.0\textwidth]{qt_range.png}
%     \caption{Блок QT GUI Range}
% \end{figure}

% Этот блок задает ползунок из QT, позволяющий удобно менять значение переменной во время работы программы.  
% Это позволяет не перезапускать программу, когда нам требуется поменять какое-либо значение.  
% Я таким образом задаю ползунок для настройки частоты приема FM волны.

% Основные параметры блока: 

% \begin{itemize}
%     \item \textbf{Default Value} — значение, которое будет устанавливаться при запуске программы;
%     \item \textbf{Start} — минимальное значение;
%     \item \textbf{Stop} — максимальное значение;
%     \item \textbf{Step} — шаг изменения при сдвиге ползунка.
% \end{itemize}

% \subsection*{\textbf{PlutoSDR Source}}

% \begin{figure}[H]
%     \centering
%     \includegraphics[width=1.0\textwidth]{source.png}
%     \caption{Блок PlutoSDR Source}
% \end{figure}

% Этот блок отвечает за приём данных от устройства ADALM-Pluto (PlutoSDR). Он подключается к SDR, управляет его настройками, получает поток отсчётов. \\

% \textbf{Параметры:}

% \begin{itemize}
%     \item \textbf{IIO context URI} — IP адрес Adalm Pluto, нужен, потому что PlutoSDR может подключаться по USB или сети (Ethernet/USB-Ethernet);
%     \item \textbf{LO (Local Oscillator)} — локальный генератор частоты или центральная частота приёма, т.е. радиостанция, которую хотим слушать;
%     \item \textbf{Sample Rate} — частота дискретизации АЦП внутри PlutoSDR. Определяет, с какой частотой будут делаться отсчеты при оцифровке;
%     \item \textbf{Buffer Size} — встроенный буфер для временного хранения данных перед их передачей в компьютер;
%     \item \textbf{Quadrature} — задаём представление сигнала в виде I/Q семплов;
%     \item \textbf{RF DC Correction} — исправляет постоянную составляющую (DC offset), которая может появляться из-за несовершенства тракта;
%     \item \textbf{BB DC Correction} — исправляет смещение в baseband-сигнале;
%     \item \textbf{Gain Mode (RX1)} — режим автоматической регулировки усиления (AGC). Slow Attack — плавное изменение усиления;
%     \item \textbf{Filter Configuration} — выбор полосовых фильтров в тракте SDR. В режиме Auto плата сама подбирает оптимальную конфигурацию фильтров;
%     \item \textbf{RF Bandwidth} — полоса пропускания приёмного тракта.
% \end{itemize}

% \subsection*{\textbf{Low Pass Filter}}

% \begin{figure}[H]
%     \centering
%     \includegraphics[width=1.0\textwidth]{filter.png}
%     \caption{Блок Low Pass Filter}
% \end{figure}

% Ограничивает полосу сигнала, выделяя только FM-станцию. \\

% \textbf{Параметры:}

% \begin{itemize}
%     \item \textbf{Decimation} — снижение частоты дискретизации в n раз;
%     \item \textbf{Gain} — усиление амплитуды после фильтрации;
%     \item \textbf{Sample Rate} — дефолтная частота дискретизации;
%     \item \textbf{Cutoff Freq} — полоса 100 кГц (ширина FM сигнала).
% \end{itemize}

% \subsection*{\textbf{QT GUI Frequency Sink}}

% \begin{figure}[H]
%     \centering
%     \includegraphics[width=1.0\textwidth]{freq_sink.png}
%     \caption{Блок QT GUI Frequency Sink}
% \end{figure}

% Этот блок при помощи QT задает спектральное представление сигнала, которое меняется в реальном времени.  
% Таких блоков 2: до фильтра (напрямую из блока source) и после фильтра. Первый показывает весь эфир, а второй — захваченный сигнал (именно FM частоту). \\

% \textbf{Параметры:}

% \begin{itemize}
%     \item \textbf{FFT Size} — кол-во точек для спектра;
%     \item \textbf{Center Frequency} — центральная частота захвата;
%     \item \textbf{Bandwidth} — полоса частот захвата.
% \end{itemize}

% \subsection*{\textbf{QT GUI Time Sink}}

% \begin{figure}[H]
%     \centering
%     \includegraphics[width=1.0\textwidth]{time_sink.png}
%     \caption{Блок QT GUI Time Sink}
% \end{figure}

% Этот блок при помощи QT задает временное представление сигнала, которое меняется в реальном времени. \\

% \textbf{Параметры:}

% \begin{itemize}
%     \item \textbf{Number of Points} — кол-во точек, отображаемых в каждый момент времени;
%     \item \textbf{Sample Rate} — частота дискретизации при отрисовке;
%     \item \textbf{Autoscale} — нужно ли масштабировать сигнал по вертикали.
% \end{itemize}

% \subsection*{\textbf{WBFM Receive}}

% \begin{figure}[H]
%     \centering
%     \includegraphics[width=1.0\textwidth]{wbfm.png}
%     \caption{Блок WBFM Receive}
% \end{figure}

% Блок демодуляции FM-сигнала. \\

% \textbf{Параметры:}

% \begin{itemize}
%     \item \textbf{Quadrature Rate} — входная частота дискретизации (после фильтра и децимации);
%     \item \textbf{Audio Decimation} — уменьшение дискретизации для звука (в моем случае до 48k, чего вполне достаточно для звука).
% \end{itemize}

% На выходе — звуковой сигнал.

% \subsection*{\textbf{Audio Sink}}

% \begin{figure}[H]
%     \centering
%     \includegraphics[width=1.0\textwidth]{audio_out.png}
%     \caption{Блок Audio Sink}
% \end{figure}

% От \textbf{WBFM Receive} звук идет на блок \textbf{Audio Sink} — блок, который выводит звуковой поток на аудиокарту хоста. \\

% \textbf{Параметры:}

% \begin{itemize}
%     \item \textbf{Sample Rate} — стандартная частота звука.
% \end{itemize}

% Соединить блоки нужно следующим образом, тогда у нас получится работающая радиосистема, которая будет принимать FM радио.

% \begin{figure}[H]
%     \centering
%     \includegraphics[width=1.0\textwidth]{system.png}
%     \caption{Пример простой радиосистемы в GNURadio}
% \end{figure}

% Пример работы программы:

% \begin{figure}[H]
%     \centering
%     \includegraphics[width=1.0\textwidth]{system.png}
%     \caption{Пример работы программы}
% \end{figure}

\chapter*{Цель и задачи}
\addcontentsline{toc}{chapter}{Цель и задачи}
\label{ch:intro}

\textbf{Цель}: Изучение системы MATLAB для инженерных,
финансовых и математических расчетов. Выполнение учебного
математического расчета \\
\\\textbf{Задачи}:

\subsection*{Часть 1. Переменные, функции, графика}

\begin{itemize}
  \item Определите функцию \(f(x)\), заданную по варианту.
  \item Постройте графики функции \(f(x)\), её производной \(f'(x)\) и интеграла
  \[
    F(x) = \int_{0}^{x} f(y)\,dy.
  \]
  \item Для вычисления производной используйте \texttt{diff}, для интеграла — \texttt{integral}.
\end{itemize}

\section*{Часть 2. Решение уравнения}

\subsection*{Условие}
\begin{itemize}
  \item Решите уравнение вида \(a\cdot x + b = f(x)\) графическим и аналитическим методами.
  \item Коэффициенты \(a\) и \(b\) задайте самостоятельно.
  \item Для численного решения используйте \texttt{fsolve}.
\end{itemize}

\section*{Часть 3. Функция двух переменных и 3D-график}

\begin{itemize}
  \item Определите функцию двух переменных \(F(x,y)\), заданную по варианту.
  \item Постройте трёхмерный график \(F(x,y)\) с использованием \texttt{meshgrid} и \texttt{surf}.
\end{itemize}

\section*{Часть 4. Матричные операции, программирование функций}
\begin{enumerate}
\item Создать новый скрипт MATLAB.
\item Разработать функцию, выполняющую следующие действия:
\begin{itemize}
\item Сгенерировать вектор-столбец, элементы которого равномерно распределены в заданном диапазоне.
\item Сгенерировать вектор-строку, элементы которого равномерно распределены в заданном диапазоне.
\item Перемножить два вектора.
\item Вычислить среднее значение и дисперсию матрицы, полученной в результате перемножения векторов.
\item В качестве аргумента функции передавать размерности векторов.
\end{itemize}
\item Произвести расчеты с помощью функции и вывести результаты.
\item Сохранить результаты работы в виде отчета.
\end{enumerate}

\section*{Часть 5. Марковские цепи. Определение и построение}
\begin{enumerate}
\item Построить сеть из L=15 узлов в виде ориентированного графа:
\begin{itemize}
\item Каждый узел должен иметь минимум 3 исходящих маршрута.
\item Каждый узел должен иметь хотя бы один входящий маршрут.
\end{itemize}
\item Построить матрицу переходов T=∥pij∥, задав:
\begin{itemize}
\item Размерность L=15
\item Ненулевые элементы вероятностей вручную, чтобы строки удовлетворяли условиям стохастичности.
\end{itemize}
\item Написать функции в MATLAB:
\begin{itemize}
\item \texttt{stochastic(matrix)}: проверяет матрицу на стохастичность.
\item \texttt{ergodic(matrix, epsilon)}: проверяет цепь Маркова на эргодичность.
\end{itemize}
\item Проверить:
\begin{itemize}
\item Стохастичность построенной матрицы.
\item Эргодичность цепи Маркова.
\end{itemize}
\end{enumerate}


\endinput
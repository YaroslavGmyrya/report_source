\documentclass[a4paper,14pt,oneside,openany]{memoir}

%%% Задаем поля, отступы и межстрочный интервал %%%

\usepackage[left=30mm, right=15mm, top=20mm, bottom=20mm]{geometry} % Пакет geometry с аргументами для определения полей
\pagestyle{plain} % Убираем стандарные для данного класса верхние колонтитулы с заголовком текущей главы, оставляем только номер страницы снизу по центру
\parindent=1.25cm % Абзацный отступ 1.25 см, приблизительно равно пяти знакам, как по ГОСТ
\usepackage{indentfirst} % Добавляем отступ к первому абзацу
%\linespread{1.3} % Межстрочный интервал (наиболее близко к вордовскому полуторному) - тут вместо этого используется команда OnehalfSpacing*

%%% Задаем языковые параметры и шрифт %%%

\usepackage[english, russian]{babel}                % Настройки для русского языка как основного в тексте
\babelfont[russian]{rm}{Times New Roman}                     % TMR в качестве базового roman-щрифта
\usepackage{float}
\usepackage{amsmath}
\usepackage{amssymb}
\usepackage{listings}
\usepackage{xcolor}
\usepackage[utf8]{inputenc}
\usepackage[T1]{fontenc}
\usepackage{minted}
%%% Задаем стиль заголовков и подзаголовков в тексте %%%

\setsecnumdepth{subsection} % Номера разделов считать до третьего уровня включительно, т.е. нумеруются только главы, секции, подсекции
\renewcommand*{\chapterheadstart}{} % Переопределяем команду, задающую отступ над заголовком, чтобы отступа не было
\renewcommand*{\printchaptername}{} % Переопределяем команду, печатающую слово "Глава", чтобы оно не печалось
\renewcommand*{\printchapternum}{} % То же самое для номера главы - тут не надо, номер главы оставляем
\renewcommand*{\chapnumfont}{\normalfont\bfseries} % Меняем стиль шрифта для номера главы: нормальный размер, полужирный
\renewcommand*{\afterchapternum}{\hspace{1em}} % Меняем разделитель между номером главы и названием
\renewcommand*{\printchaptertitle}{\normalfont\bfseries\centering\MakeUppercase} % Меняем стиль написания для заголовка главы: нормальный размер, полужирный, центрированный, заглавными буквами
\setbeforesecskip{20pt} % Задаем отступ перед заголовком секции
\setaftersecskip{20pt} % Ставим такой же отступ после заголовка секции
\setsecheadstyle{\raggedright\normalfont\bfseries} % Меняем стиль написания для заголовка секции: выравнивание по правому краю без переносов, нормальный размер, полужирный
\setbeforesubsecskip{20pt} % Задаем отступ перед заголовком подсекции
\setaftersubsecskip{20pt} % Ставим такой же отступ после заголовка подсекции
\setsubsecheadstyle{\raggedright\normalfont\bfseries}  % Меняем стиль написания для заголовка подсекции: выравнивание по правому краю без переносов, нормальный размер, полужирный

%%% Задаем параметры оглавления %%%

\addto\captionsrussian{\renewcommand\contentsname{Содержание}} % Меняем слово "Оглавление" на "Содержание"
\setrmarg{2.55em plus1fil} % Запрещаем переносы слов в оглавлении
%\setlength{\cftbeforechapterskip}{0pt} % Эта команда убирает интервал между заголовками глав - тут не надо, так красивее смотрится
\renewcommand{\aftertoctitle}{\afterchaptertitle \vspace{-\cftbeforechapterskip}} % Делаем отступ между словом "Содержание" и первой строкой таким же, как у заголовков глав
%\renewcommand*{\chapternumberline}[1]{} % Делаем так, чтобы номер главы не печатался - тут не надо
\renewcommand*{\cftchapternumwidth}{1.5em} % Ставим подходящий по размеру разделитель между номером главы и самим заголовком
\renewcommand*{\cftchapterfont}{\normalfont\MakeUppercase} % Названия глав обычным шрифтом заглавными буквами
\renewcommand*{\cftchapterpagefont}{\normalfont} % Номера страниц обычным шрифтом
\renewcommand*{\cftchapterdotsep}{\cftdotsep} % Делаем точки до номера страницы после названий глав
\renewcommand*{\cftdotsep}{1} % Задаем расстояние между точками
\renewcommand*{\cftchapterleader}{\cftdotfill{\cftchapterdotsep}} % Делаем точки стандартной формы (по умолчанию они "жирные")
\maxtocdepth{subsection} % В оглавление попадают только разделы первыхтрех уровней: главы, секции и подсекции

%%% Выравнивание и переносы %%%

%% http://tex.stackexchange.com/questions/241343/what-is-the-meaning-of-fussy-sloppy-emergencystretch-tolerance-hbadness
%% http://www.latex-community.org/forum/viewtopic.php?p=70342#p70342
\tolerance 1414
\hbadness 1414
\emergencystretch 1.5em                             % В случае проблем регулировать в первую очередь
\hfuzz 0.3pt
\vfuzz \hfuzz
%\dbottom
%\sloppy                                            % Избавляемся от переполнений
\clubpenalty=10000                                  % Запрещаем разрыв страницы после первой строки абзаца
\widowpenalty=10000                                 % Запрещаем разрыв страницы после последней строки абзаца
\brokenpenalty=4991                                 % Ограничение на разрыв страницы, если строка заканчивается переносом

%%% Объясняем компилятору, какие буквы русского алфавита можно использовать в перечислениях (подрисунках и нумерованных списках) %%%
%%% По ГОСТ нельзя использовать буквы ё, з, й, о, ч, ь, ы, ъ %%%
%%% Здесь также переопределены заглавные буквы, хотя в принципе они в документе не используются %%%

\makeatletter
    \def\russian@Alph#1{\ifcase#1\or
       А\or Б\or В\or Г\or Д\or Е\or Ж\or
       И\or К\or Л\or М\or Н\or
       П\or Р\or С\or Т\or У\or Ф\or Х\or
       Ц\or Ш\or Щ\or Э\or Ю\or Я\else\xpg@ill@value{#1}{russian@Alph}\fi}
    \def\russian@alph#1{\ifcase#1\or
       а\or б\or в\or г\or д\or е\or ж\or
       и\or к\or л\or м\or н\or
       п\or р\or с\or т\or у\or ф\or х\or
       ц\or ш\or щ\or э\or ю\or я\else\xpg@ill@value{#1}{russian@alph}\fi}
\makeatother

%%% Задаем параметры оформления рисунков и таблиц %%%

\usepackage{graphicx, caption, subcaption} % Подгружаем пакеты для работы с графикой и настройки подписей
\graphicspath{{images/}} % Определяем папку с рисунками
\captionsetup[figure]{font=small, width=\textwidth, name=Рисунок, justification=centering} % Задаем параметры подписей к рисункам: маленький шрифт (в данном случае 12pt), ширина равна ширине текста, полнотекстовая надпись "Рисунок", выравнивание по центру
\captionsetup[subfigure]{font=small} % Индексы подрисунков а), б) и так далее тоже шрифтом 12pt (по умолчанию делает еще меньше)
\captionsetup[table]{singlelinecheck=false,font=small,width=\textwidth,justification=justified} % Задаем параметры подписей к таблицам: запрещаем переносы, маленький шрифт (в данном случае 12pt), ширина равна ширине текста, выравнивание по ширине
\captiondelim{ --- } % Разделителем между номером рисунка/таблицы и текстом в подписи является длинное тире
\setkeys{Gin}{width=\textwidth} % По умолчанию размер всех добавляемых рисунков будет подгоняться под ширину текста
\renewcommand{\thesubfigure}{\asbuk{subfigure}} % Нумерация подрисунков строчными буквами кириллицы
%\setlength{\abovecaptionskip}{0pt} % Отбивка над подписью - тут не меняем
%\setlength{\belowcaptionskip}{0pt} % Отбивка под подписью - тут не меняем
\usepackage[section]{placeins} % Объекты типа float (рисунки/таблицы) не вылезают за границы секциии, в которой они объявлены

%%% Задаем параметры ссылок и гиперссылок %%% 

\usepackage{hyperref}                               % Подгружаем нужный пакет
\hypersetup{
    colorlinks=true,                                % Все ссылки и гиперссылки цветные
    linktoc=all,                                    % В оглавлении ссылки подключатся для всех отображаемых уровней
    linktocpage=true,                               % Ссылка - только номер страницы, а не весь заголовок (так выглядит аккуратнее)
    linkcolor=red,                                  % Цвет ссылок и гиперссылок - красный
    citecolor=red                                   % Цвет цитировний - красный
}

%%% Настраиваем отображение списков %%%

\usepackage{enumitem}                               % Подгружаем пакет для гибкой настройки списков
\renewcommand*{\labelitemi}{\normalfont{--}}        % В ненумерованных списках для пунктов используем короткое тире
\makeatletter
    \AddEnumerateCounter{\asbuk}{\russian@alph}     % Объясняем пакету enumitem, как использовать asbuk
\makeatother
\renewcommand{\labelenumii}{\asbuk{enumii})}        % Кириллица для второго уровня нумерации
\renewcommand{\labelenumiii}{\arabic{enumiii})}     % Арабские цифры для третьего уровня нумерации
\setlist{noitemsep, leftmargin=*}                   % Убираем интервалы между пунками одного уровня в списке
\setlist[1]{labelindent=\parindent}                 % Отступ у пунктов списка равен абзацному отступу
\setlist[2]{leftmargin=\parindent}                  % Плюс еще один такой же отступ для следующего уровня
\setlist[3]{leftmargin=\parindent}                  % И еще один для третьего уровня

%%% Счетчики для нумерации объектов %%%

\counterwithout{figure}{chapter}                    % Сквозная нумерация рисунков по документу
\counterwithout{equation}{chapter}                  % Сквозная нумерация математических выражений по документу
\counterwithout{table}{chapter}                     % Сквозная нумерация таблиц по документу

%%% Реализация библиографии пакетами biblatex и biblatex-gost с использованием движка biber %%%

\usepackage{csquotes} % Пакет для оформления сложных блоков цитирования (biblatex рекомендует его подключать)
\usepackage[%
backend=biber,                                      % Движок
bibencoding=utf8,                                   % Кодировка bib-файла
sorting=none,                                       % Настройка сортировки списка литературы
style=gost-numeric,                                 % Стиль цитирования и библиографии по ГОСТ
language=auto,                                      % Язык для каждой библиографической записи задается отдельно
autolang=other,                                     % Поддержка многоязычной библиографии
sortcites=true,                                     % Если в квадратных скобках несколько ссылок, то отображаться будут отсортированно
movenames=false,                                    % Не перемещать имена, они всегда в начале библиографической записи
maxnames=5,                                         % Максимальное отображаемое число авторов
minnames=3,                                         % До скольки сокращать число авторов, если их больше максимума
doi=false,                                          % Не отображать ссылки на DOI
isbn=false,                                         % Не показывать ISBN, ISSN, ISRN
]{biblatex}[2016/09/17]
\DeclareDelimFormat{bibinitdelim}{}                 % Убираем пробел между инициалами (Иванов И.И. вместо Иванов И. И.)
\addbibresource{bibl.bib}                           % Определяем файл с библиографией

%%% Скрипт, который автоматически подбирает язык (и, следовательно, формат) для каждой библиографической записи %%%
%%% Если в названии работы есть кириллица - меняем значение поля langid на russian %%%
%%% Все оставшиеся пустые места в поле langid заменяем на english %%%

\DeclareSourcemap{
  \maps[datatype=bibtex]{
    \map{
        \step[fieldsource=title, match=\regexp{^\P{Cyrillic}*\p{Cyrillic}.*}, final]
        \step[fieldset=langid, fieldvalue={russian}]
    }
    \map{
        \step[fieldset=langid, fieldvalue={english}]
    }
  }
}

\lstset{
    language=MATLAB,
    basicstyle=\ttfamily\small,
    keywordstyle=\color{blue},
    commentstyle=\color[HTML]{2B8315},
    stringstyle=\color{red},
    frame=single,
    breaklines=true,
    breakatwhitespace=true,
    tabsize=4,
    showstringspaces=false,
    extendedchars=true,
    inputencoding=utf8x,    
    texcl=false,
    literate=%
        {á}{{\'a}}1 {é}{{\'e}}1 {í}{{\'i}}1 {ó}{{\'o}}1 {ú}{{\'u}}1
        {Á}{{\'A}}1 {É}{{\'E}}1 {Í}{{\'I}}1 {Ó}{{\'O}}1 {Ú}{{\'U}}1
        {à}{{\`a}}1 {è}{{\`e}}1 {ì}{{\`i}}1 {ò}{{\`o}}1 {ù}{{\`u}}1
        {À}{{\`A}}1 {È}{{\'E}}1 {Ì}{{\`I}}1 {Ò}{{\`O}}1 {Ù}{{\`U}}1
        {ä}{{\"a}}1 {ë}{{\"e}}1 {ï}{{\"i}}1 {ö}{{\"o}}1 {ü}{{\"u}}1
        {Ä}{{\"A}}1 {Ë}{{\"E}}1 {Ï}{{\"I}}1 {Ö}{{\"O}}1 {Ü}{{\"U}}1
        {â}{{\^a}}1 {ê}{{\^e}}1 {î}{{\^i}}1 {ô}{{\^o}}1 {û}{{\^u}}1
        {Â}{{\^A}}1 {Ê}{{\^E}}1 {Î}{{\^I}}1 {Ô}{{\^O}}1 {Û}{{\^U}}1
        {ã}{{\~a}}1 {ẽ}{{\~e}}1 {ĩ}{{\~i}}1 {õ}{{\~o}}1 {ũ}{{\~u}}1
        {Ã}{{\~A}}1 {Ẽ}{{\~E}}1 {Ĩ}{{\~I}}1 {Õ}{{\~O}}1 {Ũ}{{\~U}}1
        {œ}{{\oe}}1 {Œ}{{\OE}}1 {æ}{{\ae}}1 {Æ}{{\AE}}1 {ß}{{\ss}}1
        {ű}{{\H{u}}}1 {Ű}{{\H{U}}}1 {ő}{{\H{o}}}1 {Ő}{{\H{O}}}1
        {ç}{{\c c}}1 {Ç}{{\c C}}1 {ø}{{\o}}1 {å}{{\r a}}1 {Å}{{\r A}}1
        {€}{{\euro}}1 {£}{{\pounds}}1 {«}{{\guillemotleft}}1
        {»}{{\guillemotright}}1 {ñ}{{\~n}}1 {Ñ}{{\~N}}1 {¿}{{?`}}1
        {¡}{{!`}}1 {°}{{\textdegree}}1 {º}{{\textordmasculine}}1
        {ª}{{\textordfeminine}}1 {£}{{\pounds}}1 {©}{{\copyright}}1
        {а}{{\selectfont\char224}}1 {б}{{\selectfont\char225}}1
        {в}{{\selectfont\char226}}1 {г}{{\selectfont\char227}}1
        {д}{{\selectfont\char228}}1 {е}{{\selectfont\char229}}1
        {ё}{{\selectfont\char168}}1 {ж}{{\selectfont\char230}}1
        {з}{{\selectfont\char231}}1 {и}{{\selectfont\char232}}1
        {й}{{\selectfont\char233}}1 {к}{{\selectfont\char234}}1
        {л}{{\selectfont\char235}}1 {м}{{\selectfont\char236}}1
        {н}{{\selectfont\char237}}1 {о}{{\selectfont\char238}}1
        {п}{{\selectfont\char239}}1 {р}{{\selectfont\char240}}1
        {с}{{\selectfont\char241}}1 {т}{{\selectfont\char242}}1
        {у}{{\selectfont\char243}}1 {ф}{{\selectfont\char244}}1
        {х}{{\selectfont\char245}}1 {ц}{{\selectfont\char246}}1
        {ч}{{\selectfont\char247}}1 {ш}{{\selectfont\char248}}1
        {щ}{{\selectfont\char249}}1 {ъ}{{\selectfont\char250}}1
        {ы}{{\selectfont\char251}}1 {ь}{{\selectfont\char252}}1
        {э}{{\selectfont\char253}}1 {ю}{{\selectfont\char254}}1
        {я}{{\selectfont\char255}}1
        {А}{{\selectfont\char192}}1 {Б}{{\selectfont\char193}}1
        {В}{{\selectfont\char194}}1 {Г}{{\selectfont\char195}}1
        {Д}{{\selectfont\char196}}1 {Е}{{\selectfont\char197}}1
        {Ё}{{\selectfont\char168}}1 {Ж}{{\selectfont\char198}}1
        {З}{{\selectfont\char199}}1 {И}{{\selectfont\char200}}1
        {Й}{{\selectfont\char201}}1 {К}{{\selectfont\char202}}1
        {Л}{{\selectfont\char203}}1 {М}{{\selectfont\char204}}1
        {Н}{{\selectfont\char205}}1 {О}{{\selectfont\char206}}1
        {П}{{\selectfont\char207}}1 {Р}{{\selectfont\char208}}1
        {С}{{\selectfont\char209}}1 {Т}{{\selectfont\char210}}1
        {У}{{\selectfont\char211}}1 {Ф}{{\selectfont\char212}}1
        {Х}{{\selectfont\char213}}1 {Ц}{{\selectfont\char214}}1
        {Ч}{{\selectfont\char215}}1 {Ш}{{\selectfont\char216}}1
        {Щ}{{\selectfont\char217}}1 {Ъ}{{\selectfont\char218}}1
        {Ы}{{\selectfont\char219}}1 {Ь}{{\selectfont\char220}}1
        {Э}{{\selectfont\char221}}1 {Ю}{{\selectfont\char222}}1
        {Я}{{\selectfont\char223}}1
}

%%% Прочие пакеты для расширения функционала %%%

\usepackage{longtable,ltcaption}                    % Длинные таблицы
\usepackage{multirow,makecell}                      % Улучшенное форматирование таблиц
\usepackage{booktabs}                               % Еще один пакет для красивых таблиц
\usepackage{soulutf8}                               % Поддержка переносоустойчивых подчёркиваний и зачёркиваний
\usepackage{icomma}                                 % Запятая в десятичных дробях
\usepackage{hyphenat}                               % Для красивых переносов
\usepackage{textcomp}                               % Поддержка "сложных" печатных символов типа значков иены, копирайта и т.д.
\usepackage[version=4]{mhchem}                      % Красивые химические уравнения
\usepackage{amsmath}                                % Усовершенствование отображения математических выражений 

%%% Вставляем по очереди все содержательные части документа %%%

\begin{document}

\thispagestyle{empty}

\begin{center}
    МИНИСТЕРСТВО ЦИФРОВОГО РАЗВИТИЯ, СВЯЗИ И МАССОВЫХ КОММУНИКАЦИЙ \\ РОССИЙСКОЙ ФЕДЕРАЦИИ

    \vspace{20pt}

    Федеральное государственное бюджетное образовательное учреждение  \\  высшего образования \\
    "<Сибирский государственный университет телекоммуникаций и информатики"> \\

    \vspace{20pt}

    Кафедра телекоммуникационных систем и вычислительных средств \\  (ТС и ВС)
\end{center}

\vfill

\begin{center}
    Отчет по лабораторной работе №5 \\  
    по дисциплине \\
    \textit{Теория массового обслуживания}

    \vspace{20pt} 
    по теме: \\
    \uppercase{Марковские цепи. Исследование эргодических свойств.
}
\end{center}

\vfill

    \noindent Студент: \\
    \textit{Группа ИА-331 \hfill Я.А Гмыря}

    \vspace{20pt}

    \noindent Предподаватель: \\
    \textit{Преподаватель \hfill А.В Андреев}

\vfill

\begin{center}
    Новосибирск 2025 г.
\end{center}                                     % Титульник

\newpage % Переходим на новую страницу
\setcounter{page}{2} % Начинаем считать номера страниц со второй
\OnehalfSpacing* % Задаем полуторный интервал текста (в титульнике одинарный, поэтому команда стоит после него)

\tableofcontents*                                   % Автособираемое оглавление

\chapter{Цель и задачи}
\label{ch:intro}

\section*{\textbf{Цель:}} 

Использование средств инженерного программного
пакета MATLAB для построения, отладки и тестирования моделей систем
массового обслуживания (СМО)

\section*{Задание к лабораторной работе}

Построить с помощью MATLAB Simulink систему массвого обслуживания M/M/1.


\endinput                                     % Введение
\chapter{Теория}
\label{ch:intro}

\section*{Введение}

На прошлом занятии мы рассматривали процесс дискретизации сигнала с помощью АЦП. На этом занятии рассмотрим обратный процесс, который
восстанавливает исходный сигнал по отсчетам.

\section*{Идеальное восстановление сигнала по отсчетам}

Восстановление сигнала по дискретным отсчетам - процесс получения непрерывного $S(t)$ по его дискретным отсчетам $S(nT_s)$.
При правильно выбранной $f_s$ (т.Котельникова) повторение спектра не накладывается в частотной области, поэтому исходный
сигнал можно получить с помощью ФНЧ:

\begin{figure}[H]
    \centering
    \includegraphics[width=1.0\textwidth]{recovery.png}
    \caption{Восстановления сигнала с помощью ФНЧ}
\end{figure} 

Спектр дискретных отсчетов сигнала в частотной области выглядит как множество копий исходного непрерывного сигнала. ФНЧ позволяет
из всех этих копий выделить только один оригинальный спектр:

\begin{figure}[H]
    \centering
    \includegraphics[width=1.0\textwidth]{recovery2.png}
    \caption{Работа ФНЧ}
\end{figure} 

Здесь $f_c$ - частота среза. Все компоненты большие $f_c$ или меньшие $-f_c$ будут подавляться. $f_c$ может принимать значения от 
$f_{max}$ до $f_s - f_{max}$, но чаще всего берут значение $f_c = \frac{f_s}{2}$. Если сигнал дискретизировался с соблюдением 
теоремы Котельникова, то $\frac{f_s}{2}$ - частота, которая как раз позволяет "захватить" оригинальный спектр.

\begin{figure}[H]
    \centering
    \includegraphics[width=1.0\textwidth]{spec.png}
    \caption{Спектр отсчетов сигнала}
\end{figure} 

Если сделать обратное преобразование над $H(i\omega)$, то получим импульсную характеристику

$$h_{LP}(t) = sinc(\pi f_st)$$

Отсюда

$$S(t) = [\sum_{n=-\infty}^{\infty}S(n)\delta(t-nT_s)] * h(t) = $$ 
$$\sum_{n=-\infty}^{\infty}S(n)h(t-nT_s)
= \sum_{n=-\infty}^{\infty}S(n)sinc(\pi f_s(t-nT_s))$$

Здесь $sinc(\pi f_s(t-nT_s))$ - ряд Котельникова. \\

Ряд Котельникова позволяет точно восстановить непрерывный сигнал.\\

Формула выше делает следующее:

\begin{figure}[H]
    \centering
    \includegraphics[width=1.0\textwidth]{plot.png}
    \caption{Восстановление сигнала}
\end{figure} 

На каждом отсчете в частотной области будет рисоваться график типа sinc(x). Таких графиков будет много и все они будут накладываться друг друга.
Сумма этих наложений и даст исходный сигнал.

\endinput                                     % Первая глава
\chapter{Практическая часть}
\label{ch:chap2}

\section*{\textbf{Передача данных между Adalm Pluto и хостом}}

Передача данных (IQ-сэмплов) между Adalm Pluto и хост-компьютером осуществляется посредством USB 2.0. Важно отметить, 
что в случае с SDR, данные необходимо передавать непрерывно в обе стороны (с хост-компьютера на SDR и обратно) одновременно. 
Хоть и теоретическая пропускная способность USB 2.0 равна 480 Mb/s, работа в полудуплексном режиме с передачей данных в 
обе стороны одновременно (с точки зрения пользователя) разительно снижается. Целевое значение частоты дискретизации 
желательно задавать в пределах ~6 Msps.

\section*{\textbf{Timestamping}}

В библиотеке SoapySDR реализованы функции получения временных меток (timestamp) с FPGA (Xilinx Zynq). Временные метки (timestamp) привязаны к каждому запросу 
данных с буфера ПЛИС, что, в свою очередь, позволяет синхронно получать/передавать данные в потоках RX/TX. Более того, 
из-за проблем с пропускной способностю USB 2.0 возникает проблема увеличения частоты дескритизации, при больших значениях которой,
USB 2.0 не может обеспечить полноценную передачу и прием (одновременных) сэмплов из Adalm Pluto на хост-компьютер. 
Выявить данную проблему можно благодаря реализации функции временных меток с Xilinx Zynq.

\section*{\textbf{Установка необходимых библиотек и зависимостей}}

\textbf{SoapySDR}

SoapySDR — открытая обобщённая API и библиотека времени выполнения для взаимодействия с SDR-устройствами. 
С помощью SoapySDR можно создавать экземпляры, настраивать и вести потоковую передачу данных с SDR-устройством в различных средах.
Большинство готовых SDR-платформ поддерживаются SoapySDR, и многие открытые приложения используют SoapySDR для интеграции с 
оборудованием. Кроме того, SoapySDR имеет привязки к средам разработки, таким как GNU Radio и Pothos.

\begin{lstlisting}
sudo apt-get install python3-pip python3-setuptools
sudo apt-get install cmake g++ libpython3-dev python3-numpy swig python3-matplotlib

git clone --branch soapy-sdr-0.8.1 https://github.com/TelecomDep/SoapySDR.git

cd SoapySDR
mkdir build && cd build

cmake ../

make -j 16
sudo make install
sudo ldconfig
\end{lstlisting}

\textbf{Libiio}

libiio — библиотека, разработанная компанией Analog Devices, которая предназначена для упрощения работы с устройствами ввода-вывода
данных (I/O), особенно с программируемыми аналогово-цифровыми и цифро-аналоговыми преобразователями (ADC/DAC), а также с 
радиооборудованием на базе платформы ADI (например, ADALM-PLUTO). Позволяет читать и записывать данные в реальном времени.

\begin{lstlisting}
sudo apt-get install libxml2 libxml2-dev bison flex libcdk5-dev cmake
sudo apt-get install libusb-1.0-0-dev libaio-dev pkg-config 
sudo apt install libavahi-common-dev libavahi-client-dev


git clone --branch v0.24 https://github.com/TelecomDep/libiio.git

cd libiio
mkdir build && cd build
cmake ../
make -j 16
sudo make install
\end{lstlisting}

\textbf{LibAD9361}

LibAD9361 - библиотека для работы с радиочипами семейства AD9361 от Analog Devices. В сочетании с libiio позволяет организовать 
потоковое чтение/запись данных в реальном времени.

\begin{lstlisting}
git clone --branch v0.3 https://github.com/TelecomDep/libad9361-iio.git
cd libad9361-iio

mkdir build && cd build

cmake ../

make -j 16
sudo make install
sudo ldconfig
\end{lstlisting}

\textbf{SoapyPlutoSDR}

SoapyPlutoSDR - библиотека, которая является расширением библиотеки SoapySDR, предназначенная для работы конкретно с Adalm Pluto. 

\begin{lstlisting}
git clone --branch sdr_gadget_timestamping https://github.com/TelecomDep/SoapyPlutoSDR.git
cd SoapyPlutoSDR

mkdir build && cd build

cmake ../

make -j 16
sudo make install
sudo ldconfig
\end{lstlisting}

\section*{\textbf{Основне моменты работы с Adalm Pluto напрямую из C++}}

\subsection*{\textbf{Подключение библиотек}}

\begin{lstlisting}
// Init device
#include <SoapySDR/Device.h>  
// Data types for writing samples 
#include <SoapySDR/Formats.h>  
\end{lstlisting}

\subsection*{\textbf{Инициализация устройства}}

\begin{lstlisting}
//create struct for init
SoapySDRKwargs args = {};

//Select device type
SoapySDRKwargs_set(&args, "driver", "plutosdr");       
if (1) {
    // Sample transmission method (usb)
    SoapySDRKwargs_set(&args, "uri", "usb:");           
} else {
    // Or IP
    SoapySDRKwargs_set(&args, "uri", "ip:192.168.2.1"); 
}
SoapySDRKwargs_set(&args, "direct", "1");
// Buffer size and timestamps              
SoapySDRKwargs_set(&args, "timestamp_every", "1920");  
// Use antennas?
SoapySDRKwargs_set(&args, "loopback", "0");
// Init             
SoapySDRDevice *sdr = SoapySDRDevice_make(&args);    
// Free memory   
SoapySDRKwargs_clear(&args);
\end{lstlisting}

\subsection*{\textbf{Формирование потоков и буферов}}

\begin{lstlisting}
// create streams
SoapySDRStream *rxStream = SoapySDRDevice_setupStream(sdr, SOAPY_SDR_RX, SOAPY_SDR_CS16, channels, channel_count, NULL);
SoapySDRStream *txStream = SoapySDRDevice_setupStream(sdr, SOAPY_SDR_TX, SOAPY_SDR_CS16, channels, channel_count, NULL);

//start streaming
SoapySDRDevice_activateStream(sdr, rxStream, 0, 0, 0); 
SoapySDRDevice_activateStream(sdr, txStream, 0, 0, 0);


// Get RX/TX MTU sizes

size_t rx_mtu = SoapySDRDevice_getStreamMTU(sdr, rxStream);
size_t tx_mtu = SoapySDRDevice_getStreamMTU(sdr, txStream);

// allocate memory for buffers (for RX/TX samples)
int16_t tx_buff[2 *tx_mtu];
int16_t rx_buffer[2 *rx_mtu];
\end{lstlisting}


\subsection*{\textbf{Получение I/Q семплов}}

\begin{lstlisting}
// start receive samples
for (size_t buffers_read = 0; buffers_read < iteration_count; buffers_read++)
{
    void *rx_buffs[] = {rx_buffer};
    // flags set by receive operation
    int flags;    
    //timestamp for receive buffer   
    long long timeNs; 
    
    // Read samples from stream and write I/Q samples in file
    int sr = SoapySDRDevice_readStream(sdr, rxStream, rx_buffs, rx_mtu, &flags, &timeNs, timeoutUs);
    // write in file
    for(int i = 0; i < rx_mtu * 2; i++){
        fprintf(file, "%d %d\n", rx_buffer[i], rx_buffer[i+1]);
    }
}
\end{lstlisting}

\subsection*{\textbf{Освобождение памяти}}

\begin{lstlisting}
//stop streaming
SoapySDRDevice_deactivateStream(sdr, rxStream, 0, 0);
SoapySDRDevice_deactivateStream(sdr, txStream, 0, 0);

//shutdown the stream
SoapySDRDevice_closeStream(sdr, rxStream);
SoapySDRDevice_closeStream(sdr, txStream);

//cleanup device handle
SoapySDRDevice_unmake(sdr);
\end{lstlisting}

После работы программы создатся файл samples.txt, в котором будут храниться полученные семплы в формате: (I,Q). Для визуализации
I(t) и Q(t) напишем простой парсер на Python. Результат:

\begin{figure}[H]
    \centering
    \includegraphics[width=0.8\textwidth]{ItQt.png}
    \caption{Графики I(t) и Q(t)}
\end{figure}

Можем наблюдать, что графики напонимают прямоугольный сигнал.

\endinput
                                     % Вторая глава                                 % Четвертая глава
\chapter{Часть 5. Марковские цепи. Определение и построение}
\label{ch:сhap5}

Реализация:\\

\begin{figure}[ht]
    \centering
    \includegraphics[width=1.0\textwidth]{IA_331_lab1_1_3p1.png}
    \caption{Реализация заданий из части 5}
    \label{fig:open_audio}
\end{figure}

\begin{figure}[ht]
    \centering
    \includegraphics[width=1.0\textwidth]{IA_331_lab1_1_3p2.png}
    \caption{Реализация заданий из части 5}
    \label{fig:open_audio}
\end{figure}

Результат:

\begin{figure}[ht]
    \centering
    \includegraphics[width=1.0\textwidth]{IA_331_lab1_3_1_res.png}
    \caption{Результат работы программы для части 5}
    \label{fig:open_audio}
\end{figure}


\endinput  

\printbibliography[title=Список использованных источников] % Автособираемый список литературы

\end{document}
\chapter{Дополнительные задачи. Оценка теоретического и реального математического ожидания}
\label{ch:chap3}

Теоретическое математическое ожидание для непрерывной велчины - интеграл от плотности распределения на определенном интервале (всё задано в таблице)

Вычисление:

\textbf{Математическое ожидание для непрерывной величины:}
\[
E[X] = \int_0^{1.5} x f(x) \, dx
= \int_0^{1.5} x \cdot \frac{3}{2} \frac{x^{1/2}}{(1.5)^{3/2}} \, dx
= \frac{3}{2 (1.5)^{3/2}} \int_0^{1.5} x^{3/2} \, dx
\]

\[
\int_0^{1.5} x^{3/2} dx = \frac{2}{5} x^{5/2} \Big|_0^{1.5} = \frac{2}{5} (1.5)^{5/2}
\]

\[
E[X] = \frac{3}{2 (1.5)^{3/2}} \cdot \frac{2}{5} (1.5)^{5/2} = \frac{3}{5} \cdot 1.5 = 0.9
\]

\textbf{Математическое ожидание для дискретной величины:}
Теоретическое математическое ожидание для дискретной велчины - сумма произведений элементов выборки на вероятность их "выпадения".
В моем случае используется геометрическое случайное распределение, поэтому формула упрощается до:

\[
E[X] = \frac{1-p}{p}, \quad \text{где } p = 0.1
\]

\[
E[X] = \frac{1 - 0.1}{0.1} = \frac{0.9}{0.1} = 9
\]


Можно сравнить с результатами, полученными ранее. Теоретическое и экспериментальное математическое ожидание отличаются на допустимые значения.


\textbf{Дисперсия для непрерывной величины:}
\[
E[X^2] = \int_0^{1.5} x^2 f(x) \, dx
= \int_0^{1.5} x^2 \cdot \frac{3}{2} \frac{x^{1/2}}{(1.5)^{3/2}} \, dx
= \frac{3}{2 (1.5)^{3/2}} \int_0^{1.5} x^{5/2} \, dx
\]

\[
\int_0^{1.5} x^{5/2} dx = \frac{2}{7} x^{7/2} \Big|_0^{1.5} = \frac{2}{7} (1.5)^{7/2}
\]

\[
E[X^2] = \frac{3}{2 (1.5)^{3/2}} \cdot \frac{2}{7} (1.5)^{7/2} = \frac{3}{7} (1.5)^2 \approx 0.9643
\]

\[
D[X] = E[X^2] - (E[X])^2 = 0.9643 - 0.9^2 \approx 0.1543
\]

\textbf{Итог:}
\[
\boxed{E[X] = 0.9, \quad D[X] \approx 0.1543}
\]

\textbf{Дисперсия для дискретной величины:}

\[
D[X] = \sum_{k=0}^{\infty} (k - E[X])^2 \, p (1-p)^k = \frac{1-p}{p^2}
\]

\[
D[X] = \frac{1-0.1}{(0.1)^2} = \frac{0.9}{0.01} = 90
\]


\endinput
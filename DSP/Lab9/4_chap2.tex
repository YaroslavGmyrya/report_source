\chapter{Практика}
\label{ch:chap2}

\section*{Задание 1: растчет ИХ}

Расчитаем импульсную характеристику идеального ФНЧ по формуле $H_02f_csinc(w_c(t-t_0))$. Будем придавать $f_c$ следующие значения: 100,200,400 Гц.

\begin{figure}[H]
    \centering
    \includegraphics[width=1.0\textwidth]{imp_resp.png}
    \caption{Импульсная характеристика идеального ФНЧ}
\end{figure} 

Графики расположены в порядке возрастания частоты среза. Можем заметить, что длина импульсной характеристики (расстояние до первого нуля)
обратно пропорционально зависит от частоты среза $f_c$. Чем больше частота среза, тем меньше длина импульсной характеристики.

\subsection*{Задание 2: Вычисление частотной характеристики ФНЧ}

Численно выполним ППФ над импульсной характеристикой и получим $H(i\omega)$. Вычислим АЧХ ФНЧ как $abs(H(i\omega))$ и ФЧХ, как 
$\phi(\omega) = -\omega*t_0$ 

\begin{figure}[H]
    \centering
    \includegraphics[width=1.0\textwidth]{APR.png}
    \caption{АЧХ и ФЧХ ФНЧ}
\end{figure}

По графику АЧХ видно, что частота среза составляет примерно 115Гц. Из графика ФЧХ видно, что угол наклона равен $-t_0$.

\subsection*{Задание 3: Дискретизация и восстановление сигналов по отсчетам. Явление подмены сигнала при восстановлении}

Изобразим сигнал $s(t) = cos(2\pi ft)$ при $f$ = 2Гц, $t \in [0;4]$. Далее произведем дискретизацию сигнала с шагом $T_s$ = 0.4:

\begin{figure}[H]
    \centering
    \includegraphics[width=1.0\textwidth]{3_1.png}
    \caption{Восстановление сигнала}
\end{figure}

Частота дискретизации составляет $\frac{1}{0.4}$ = 2.5Гц, что достаточно мало и не удовлетворяет теореме Котельникова. Дискретный
сигнал мало похож на исходный сигнал и по этой причине восстановить его не удалось. \\

Выполним те же действия но при $T_s$ = 0.2:

\begin{figure}[H]
    \centering
    \includegraphics[width=1.0\textwidth]{3_2.png}
    \caption{Восстановление сигнала}
\end{figure}

Теперь частота дискретизации равна 5Гц, и теорема Котельникова выполняется. График дискретного сигнала уже больше похож на исходный, 
т.к мы взяли больше отсчетов. Восстановленный сигнал совпадает с оригиналом. \\

Изобразим на бумаге спектры двух дискретных сигналов: с $T_s$ = 0.4 и $T_s$ = 0.2:

\begin{figure}[H]
    \centering
    \includegraphics[width=1.0\textwidth]{3_3.png}
    \caption{Спектры дискретных сигналов}
\end{figure} 

Можем заметить, что в первом в случае в полосу пропускания фильтра попали посторонние компоненты (должна быть только нулевая). На втором
графике после уменьшения $T_s$ вдвое частота дискретизации стала в 2 раза больше и теперь в область пропускания фильтра попадает
только одна основная компонента в нуле.

\section*{Задание 4: Визуализировать процесс восстановления подробнее}

Визуализируем процесс восстановления более подробно: нанесем на график не сумму членов ряда Котельникова, а каждый член ряда.

\begin{figure}[H]
    \centering
    \includegraphics[width=1.0\textwidth]{4_1.png}
    \caption{Ряд Котельникова}
\end{figure}

Получили множество повторяющихся графиков, каждый из которых напоминает идеальную импульсную характеристику ФНЧ.
\begin{figure}[H]
    \centering
    \includegraphics[width=1.0\textwidth]{4_2.png}
    \caption{Ряд Котельникова}
\end{figure}

Вместе графики образуют что-то похожее на огибающую сигнала. Если графики проинтегрировать (просуммировать в нашем случае), то получим
график исходного сигнала.


\endinput 
\chapter{Теория}
\label{ch:intro}

\section*{Введение}

На прошлом занятии мы рассматривали процесс дискретизации сигнала с помощью АЦП. На этом занятии рассмотрим обратный процесс, который
восстанавливает исходный сигнал по отсчетам.

\section*{Идеальное восстановление сигнала по отсчетам}

Восстановление сигнала по дискретным отсчетам - процесс получения непрерывного $S(t)$ по его дискретным отсчетам $S(nT_s)$.
При правильно выбранной $f_s$ (т.Котельникова) повторение спектра не накладывается в частотной области, поэтому исходный
сигнал можно получить с помощью ФНЧ:

\begin{figure}[H]
    \centering
    \includegraphics[width=1.0\textwidth]{recovery.png}
    \caption{Восстановления сигнала с помощью ФНЧ}
\end{figure} 

Спектр дискретных отсчетов сигнала в частотной области выглядит как множество копий исходного непрерывного сигнала. ФНЧ позволяет
из всех этих копий выделить только один оригинальный спектр:

\begin{figure}[H]
    \centering
    \includegraphics[width=1.0\textwidth]{recovery2.png}
    \caption{Работа ФНЧ}
\end{figure} 

Здесь $f_c$ - частота среза. Все компоненты большие $f_c$ или меньшие $-f_c$ будут подавляться. $f_c$ может принимать значения от 
$f_{max}$ до $f_s - f_{max}$, но чаще всего берут значение $f_c = \frac{f_s}{2}$. Если сигнал дискретизировался с соблюдением 
теоремы Котельникова, то $\frac{f_s}{2}$ - частота, которая как раз позволяет "захватить" оригинальный спектр.

\begin{figure}[H]
    \centering
    \includegraphics[width=1.0\textwidth]{spec.png}
    \caption{Спектр отсчетов сигнала}
\end{figure} 

Если сделать обратное преобразование над $H(i\omega)$, то получим импульсную характеристику

$$h_{LP}(t) = sinc(\pi f_st)$$

Отсюда

$$S(t) = [\sum_{n=-\infty}^{\infty}S(n)\delta(t-nT_s)] * h(t) = $$ 
$$\sum_{n=-\infty}^{\infty}S(n)h(t-nT_s)
= \sum_{n=-\infty}^{\infty}S(n)sinc(\pi f_s(t-nT_s))$$

Здесь $sinc(\pi f_s(t-nT_s))$ - ряд Котельникова. \\

Ряд Котельникова позволяет точно восстановить непрерывный сигнал.\\

Формула выше делает следующее:

\begin{figure}[H]
    \centering
    \includegraphics[width=1.0\textwidth]{plot.png}
    \caption{Восстановление сигнала}
\end{figure} 

На каждом отсчете в частотной области будет рисоваться график типа sinc(x). Таких графиков будет много и все они будут накладываться друг друга.
Сумма этих наложений и даст исходный сигнал.

\endinput
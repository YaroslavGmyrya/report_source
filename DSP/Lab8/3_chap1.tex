\chapter{Теория}
\label{ch:intro}

\section*{Введение}

На предыдущих занятиях мы говорили только об аналоговых сигналах. От них никуда не деться и в радиоканале сигнал 
распространяется именно как аналоговый сигнал, но современная техника не может хранить аналоговые величины из-за технических
особенностей, поэтому осуществляется переход от аналогового сигнала к дискретному. С этого занятия начнем изучение именно дискретных
сигналов.

\section*{АЦП и ЦАП}

Дискретизация (Аналогово-цифровое преобразование) - процесс получения значений непрерывного сигнала в дискретные моменты времени. \\

АЦП (ADC) - электронное устройство, которое преобразует непрерывный аналоговый сигнал в набор дискретных отсчетов (samples). Помимо
дискретизации АЦП выполняет еще одну важную функцию: квантование. Квантование - сопоставление значению сигнала набора бит. Сами
семплы по большому счету это до сих пор аналоговый сигнал, т.е у нас имеется только набор значений сигнала в дискретные моменты времени,
но это еще не нули и единицы.

\begin{figure}[H]
    \centering
    \includegraphics[width=1.0\textwidth]{adc.png}
    \caption{Пример дискретизации сигнала}
\end{figure}

Расстояние между отсчетами во времени - интервал дискретизации $T_s$. Число отсчетов произведенных за 1 секунду - частота дискретизации
$f_s = \frac{1}{T_s}$. \\

Отсчет в момент времени n равен:

$$S[n] = S(nT_s)$$

Процесс, обратный дискретизации - восстановление непрерывного сигнала по отсчетам (ЦАП - Цифро-аналоговое преобразование (DAC)).

\section*{Теорема Котельникова}

Важно верно выбрать $f_s$. При слишком большой $f_s$ получим избыток отсчетов, которые будут нагружать систему, а результат не станет
сильно лучше. При малой $f_s$ происходит потеря информации о сигнале. \\

Правила выбора $f_s$ описываются теоремой Котельникова (теорема семплов, теорема Найквиста, теорема Шеннона). \\

Для точного и полного описания непрерывного сигнала его дискретными отсчетами $f_s$ выбирается так:

$$\boxed{f_s \ge 2f_{max}}$$

Здесь $f_{max}$ - частота максимальной компоненты в спектре сигнала. \\

При выполнении условия выше можно точно восстановить непрерывный сигнал из отсчетов. \\

\section*{Явление подмены}

Следствие неправильного выбора $f_s$ (aliasing), при котором отсчеты описывают не исходный сигнал, а какой-то другой, при этом
оба эти сигнала становятся неразличимы.

\section*{Частота Найквиста}

Наибольшая частота аналогового сигнала, которую можно корректно восстановить из его дискретных отсчётов.

$$\boxed{F_N = \frac{f_s}{2}}$$

\section*{Идеальный дискретизатор. Спектр дискретных отсчетов сигнала}

Идеальный дискретизатор можно описать как перемножитель сигналов:

\begin{figure}[H]
    \centering
    \includegraphics[width=1.0\textwidth]{ideal_adc.png}
    \caption{Схема идеального дискретизатора}
\end{figure}

Здесь S(t) - непрерывный сигнал, d(t) - набор поочередно включающихся импульсов (дельта функций) со сдвигом $nT_s$. Путем такого перемножения мы можем получить
значение сигнала в момент $nT_s$.

\begin{figure}[H]
    \centering
    \includegraphics[width=1.0\textwidth]{example.png}
    \caption{Пример работы дискретизатора}
\end{figure}

$$S_d(t) = S(t) * \sum_{n = -\infty}^{\infty}\delta(t - nT_s) = \sum_{n = -\infty}^{\infty}S(t)\delta(t - nT_s) = \boxed{\sum_{n = -\infty}^{\infty} S(nT_s)\delta(t - nT_s)}$$

Получили формулу дискретной свертки. \\

По свойству ПФ перемножение двух сигналов во времени соответствует свертке их спектров:

$$S_d(f) = S(f) * FT[\sum_{n = -\infty}^{\infty}\delta(t-nT_s)] = S(f) * \frac{1}{T_s}\sum_{k = -\infty}^{\infty}\delta(f-kf_s) = $$

$$\frac{1}{T_s}\sum_{k = -\infty}^{\infty}S(f-kf_s)$$

Здесь S(f) - спектр S(t). \\

Спектр дискретных отсчетов - сумма повторений исходного спектра с периодом равным $f_s$:

\begin{figure}[H]
    \centering
    \includegraphics[width=1.0\textwidth]{spec1.png}
    \caption{Спектр дискретного сигнала}
\end{figure}

Предположим, что $f_s$ < 2$f_max$, тогда расстояние между повторениями исходного спектра будет слишком маленьким и они будут
накладываться друг на друга:

\begin{figure}[H]
    \centering
    \includegraphics[width=1.0\textwidth]{spec2.png}
    \caption{Иллюстрация теоремы Котельникова}
\end{figure}

В итоге семплы будут описывать какой-то другой сигнал, а не исходный.


\endinput
% \chapter{Практика}
% \label{ch:chap2}

% \subsection*{Задание 1: вычисление интеграла свертки сигнала и импульсной характеристики}

% Зададим гармонический сигнал и импульсную характеристику RC цепи. Выполним операцию свертки и получим выходной сигнал.

% \begin{figure}[H]
%     \centering
%     \includegraphics[width=1.0\textwidth]{task1.png}
%     \caption{Задание 1}
% \end{figure} 

% Можем заметить, что амплитуда кратно увеличилась, фаза тоже изменилась.

% \subsection*{Задание 2: вычисление частотной характеристики RC цепи}

% Вычислим АЧХ и ФЧХ RC цепи:

% \begin{figure}[H]
%     \centering
%     \includegraphics[width=1.0\textwidth]{task2.png}
%     \caption{АЧХ и ФЧХ RC цепи}
% \end{figure}

% \subsection*{Задание 3: вычисление сигнала на выходе линейной цепи по частотной характеристике цепи}

% Зная АЧХ и ФЧХ цепи, мы можем вычислить значение выходного сигнала. Наш сигнал имеет частоту 2000 Гц. Если посмотреть на график
% АЧХ и ФЧХ, то этому значению соответствуют амплитуда 0.45В и фаза -1.9 рад. Выходной сигнал будет иметь вид $0.45cos(4000\pi+1.9)$. 

% \begin{figure}[H]
%     \centering
%     \includegraphics[width=1.0\textwidth]{out_sig.png}
%     \caption{Выходной сигнал}
% \end{figure}

% Видим, что в выходном сигнале изменилась фаза и амплитуда, но форма сигнала осталась прежней.

% \endinput
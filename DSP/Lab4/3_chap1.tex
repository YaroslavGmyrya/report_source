\chapter{Теория}
\label{ch:intro}

\section*{\textbf{Введение}}

На прошлых занятиях мы рассматривали ряд Фурье и разные варианты его записи: тригонометрическая, косинусная, комплексная формы. Но все эти варианты записи
объединяло то, что они работали только с периодическими сигналами. На этом занятии рассмотрим преобразование Фурье, которое приминимо к непериодическим сигналам.

\section*{\textbf{Ряд Фурье в комплексной форме}}

Периодический сигнал можно представить в виде ряда Фурье. Совокупность гармоник ряда Фурье - спектр сигнала. Спектр сигнала дискретный (линейчатый), состоит из 
спектральных линий частот, кратных $\omega_0$.

Для непериодического сигнала $T \rightarrow \infty$. При увеличении периода сигнала интервалы между спектр. линиями уменьшаются. Пусть $T \rightarrow \infty$, 
тогда спектр линий сливается, образуя сплошной непрерывный спектр - спектральную функцию.

\[
S(t) = \sum_{n=-\infty}^{\infty} C_n e^{-i\omega_0nt}
\]

\[
C_n = \frac{1}{T} \int_{-\frac{T}{2}}^{\frac{T}{2}} S(t)e^{-i\omega_0nt} \, dt
\]

Подставим $C_n$ и заменим $\frac{1}{T}$ на $\frac{\omega_0}{2\pi}$:

\[
S(t) = \sum_{n=-\infty}^{\infty} \left[ \frac{\omega_0}{2\pi} \int_{-\frac{T}{2}}^{\frac{T}{2}} S(t)e^{-i\omega_0nt} \, dt \right] e^{in\omega_0t}
\]

Какой сигнал можно назвать непериодическим? Тот, у которого период бесконечно большой, т.е T->$\infty$. При \(T \to \infty\) частота первой гармоники 
\(\omega_1 = \dfrac{2\pi}{T}\) уменьшается и гармоники располагаются чаще, спектр вместо дискретного становится сплошным, непрерывным. Амплитуды отдельных гармоник 
становятся бесконесно малыми.


Пусть $T \rightarrow \infty$, тогда $\omega_0 \rightarrow\delta \omega \to 0$, $n\omega_0 \rightarrow \omega$

Тогда ряд Фурье запишем как:

\[
S(t) = \frac{1}{2\pi} \lim_{\delta \omega \rightarrow 0} \sum_{n=-\infty}^{\infty} \left[ \int_{-\infty}^{\infty} S(t)e^{-i\omega t} \, dt \right] e^{i\omega t} \delta \omega
\]

Сумма переходит в пределе в интеграл.

\[
S(t) = \frac{1}{2\pi} \int_{-\infty}^{+\infty} \left[ \int_{-\infty}^{+\infty} S(t)e^{i\omega t} \, dt \right] e^{i\omega t} \, d\omega
\]

Преобразование Фурье:

\[
S(i\omega) = \int_{-\infty}^{\infty} S(t)e^{-i\omega_0t} \, dt \quad \rightarrow \text{ПФ}
\]

Обратное ПФ:

\[
S(t) = \frac{1}{2\pi} \int_{-\infty}^{\infty} S(i\omega)e^{i\omega_0t} \, d\omega \quad - \text{ОПФ}
\]

$S(i\omega)$ соответствует комплексному коэффициенту ряда Фурье.

Прямое преобразование Фурье преобразует сигнал из временной области в частотную область, т.к $T \to \infty$, то коэффициент $C_n$ соответствует конечной амплитуде в 
узкой полосе частот $d\omega$. $S(i\omega)$ - комплексная функция, т.е $S(i\omega) = |S(i\omega)|e^{i\phi\omega}$. \\

$|S(i\omega)| = \sqrt{Re[S(i\omega)]^2 + Im[S(i\omega)]^2}$ - амплитудный спектр \\

$\phi(w) = -arctg\frac{Im[S(i\omega)]}{Re[S(i\omega)]}$ - фазовый спектр.


\end{document}

\endinput